\section{Manual do Usuário}

Este manual serve para o administrador do SiGE entender como funciona as
telas do usuário. Pretendemos mostrar cada tela e identificar sua
função.

\subsection{Telas de acesso de usuários deslogados}

São telas de acesso público, onde qualquer usuário pode acessar.

\subsubsection{Página inicial \label{home}}

Mapeada como: \texttt{/ {[}application/IndexController.index{]}}

Primeira página do sistema, contendo o banner, links para a inscrição
\ref{inscricao} e login \ref{login}, além de contador regressivo para o
evento.

\subsubsection{Login \label{login}}

Mapeada como: \texttt{/login {[}application/IndexController.login{]}}

Página comum de login, onde o e-mail e a senha são requeridos. Contém
links para Esqueci minha senha \ref{esqueci-senha} e inscrição
\ref{inscricao}.

\subsubsection{Inscrição \label{inscricao}}

Mapeada como :
\texttt{/participar {[}application/ParticipanteController.criar{]}}

Página para preenchimento dos dados necessário para se cadastrar. Logo
após o cadastro um e-mail é enviado para o usuário contendo alguns dados
dele, além de uma senha gerada automaticamente pelo sistema.

\paragraph{OBSERVAÇÃO:}

esse sistema de enviar a senha deve mudar em breve, pois isso pode
causar um problema de segurança.

\subsubsection{Esqueci minha senha \label{esqueci-senha}}

Mapeada como:
\texttt{/recuperar-senha {[}application/IndexController.recuperar-senha{]}}

Nessa tela apenas o e-mail é requerido. Logo em seguida uma nova senha
gerada pelo sistema é enviada para o e-mail do usuário.

\paragraph{OBSERVAÇÃO:}

essa tela deve mudar muito. Pedir apenas o e-mail pode gerar ataque de
DoS, troca de senhas não desejadas, etc.

Para começar teremos um Captcha, além do pedido de outros dados.

\subsubsection{Sobre \label{sobre}}

Mapeada como: \texttt{/sobre {[}application/IndexController.sobre{]}}

Página com os envolvidos no projeto, tecnologias utilizadas.

\subsubsection{Programação \label{programacao}}

Mapeada como:
\texttt{/programacao {[}application/EventoController.programacao{]}}

Página com toda a programação do evento, com detalhes sobre o
palestrante, horário, conteúdo abordado, etc.

\subsubsection{Detalhes do evento \label{evento}}

Mapeada como: \texttt{/e/:id {[}application/EventoController.ver{]}}

Página com todos os detalhes do evento, se é palestra, mini-curso ou
oficina, o horário, nível, links para compartilhar em redes sociais e
comentários usando a plataforma \href{http://disqus.com/}{disqus}.

\paragraph{NOTA:}

usamos a palavra \textbf{Encontro} para definir um conjunto de eventos.
\textbf{Evento} é definido como uma palestra, mini-curos ou oficina.

\subsubsection{Página do usuário \label{usuario}}

Mapeada como:
\texttt{/u/:id {[}application/ParticipanteController.ver{]}}

Cada usuário tem uma página pública onde é mostrado alguns detalhes
sobre ele como bio, twitter, apresentações do slideshare.

\paragraph{OBSERVAÇÃO:}

devido a uma mudança no RSS do slideshare, alguns usuários apresentam
problemas com a visualização das suas apresentações mesmo colocando o
usuário do slideshare corretamente. Para isso precisamos usar a nova
\href{http://apiexplorer.slideshare.net/}{api}.

\subsection{Telas de acesso de usuários logados}

\subsubsection{Participante \label{participante}}

Mapeada como:
\texttt{/participante {[}application/ParticipanteController.index{]}}

Tela inicial logo após login, tanto para usuário comum quanto para
administrador. Nela temos uma lista de interesses, bem como o link para
ver e adicionar interesse \ref{interesse} em palestras, mini-cursos ou
oficinas.

\subsubsection{Interesse em Eventos \label{interesse}}

Mapeado como:
\texttt{/evento/interesse {[}application/EventoController.interesse{]}}

Tela de visualização e busca de eventos do interesse do participante,
para que ele possa marcar e depois ver uma lista somente com os eventos
que tem interesse (ver \ref{participante}).

É possível ver eventos por dia e por tipo, além de ver o horário. Assim
o participante vai para o que mais se adequada ao seu tempo no Encontro.

\paragraph{OBSERVAÇÃO:}

Marcar um evento como interesse não garante a vaga do participante na
palestra. Em outra versão do SiGE será possível ver a demanda e
selecionar uma sala adequada ao tamanho do evento.

\subsubsection{Caravana \label{caravana}}

Mapeada como:
\texttt{/caravana {[}application/CaravanaController.index{]}}

Esta tela é para quem quer ser responsável por uma caravana. Basta
clicar em Criar Caravana \ref{criar-caravana} e preencher os dados e
definir os participantes que farão parte da Caravana.

\paragraph{NOTA:}

Esta tela serve principalmente de controle de homens e mulheres que vem
na Caravana, ainda mais quando eles tem que passar mais de um dia. Assim
é possível separar os homens das mulheres à noite e arrumar locais
condizentes com o número de pessoas do mesmo gênero.

\paragraph{NOTA 2:}

Ao ser adicionado numa caravana, o participante ver os dados de sua
caravana junto com um link caso queira sair. Pode ser porque decidiu não
ir mais, ou foi colocado por engano.

\subsubsection{Criar Caravana \label{criar-caravana}}

Mapeada como:
\texttt{/caravana/criar {[}application/CaravanaController.criar{]}}

Nessa tela o responsável pela caravana preenche os dados requeridos.
Logo após ele é redirecionado a tela da Caravana \ref{caravana}, onde
agora é possível adicionar os participantes
\ref{participantes-caravana}.

\subsubsection{Participantes da Caravana \label{participantes-caravana}}

Mapeada como: \texttt{/caravana/participantes}

\texttt{{[}application/CaravanaController.participantes{]}}

Tela acessada somente pelo Responsável pela Caravana, onde é possível
adicionar os participantes da caravana através de seu endereço de
e-mail.

\paragraph{OBSERVAÇÃO:}

O endereço de e-mail aparecerá somente se o participantes estiver
cadastrado no Encontro atual, mesmo que ele tenha participado de outros.

\subsubsection{Editar Caravana \label{editar-caravana}}

Mapeada como:
\texttt{/caravana/editar {[}application/CaravanaController.editar{]}}

Tela para edição dos dados da Caravana, acessada somente pelo
Responsável pela Caravana.

\subsubsection{Submissão de trabalhos \label{submissao}}

Mapeado como:
\texttt{/submissao {[}application/EventoController.index{]}}

Página que lista os trabalhos submetidos, bem como um link para submeter
um trabalho \ref{submeter}.

\subsubsection{Submeter trabalho \label{submeter}}

Mapeado como:
\texttt{/evento/submeter {[}application/EventoController.submeter{]}}

Tela para preenchimento dos dados do trabalho (ou evento).

\subsubsection{Editar evento \label{editar-evento}}

Mapeado como:
\texttt{/evento/editar/id/:id {[}application/EventoController.editar{]}}

Tela para edição dos dados do evento.

\paragraph{OBSERVAÇÃO:}

Tanto a tela Submeter Trabalho \ref{submeter} quanto Editar evento
\ref{editar-evento} só podem ser acessadas no período de submissão de
trabalho definido pelo Administrador ao criar o evento. Apenas um
administrador pode submeter trabalho após o período de submissão.

\subsubsection{Adicionar outros palestrantes \label{add-palestrantes}}

Mapeada como: \texttt{/evento/outros-palestrantes/id/:id}

\texttt{{[}application/EventoController.outros-palestrantes{]}}

Alguns eventos podem demandar mais de uma pessoa para ser realizado.
Esse é o propósito dessa tela, assim todos os envolvidos recebem
certificado de palestrante.

\subsubsection{Criar / Editar tags \label{tags}}

Mapeada como:
\texttt{/evento/tags/id/:id {[}application/EventoController.tags{]}}

Adiciona e edita tags do evento. As tags indicam quais tecnologias a
serem utilizadas, ou ferramentas específicas, sistema, etc.

\paragraph{OBSERVAÇÃO:}

É possível ver as tags nos Detalhes do evento \ref{evento} mas ainda não
é possível fazer buscas ou qualquer outro uso. Isso será feito em uma
outra versão do SiGE.

\subsubsection{Página pessoal do usuário \label{pagina-pessoal}}

Mapeada como:
\texttt{/participante/ver {[}application/ParticipanteController.ver{]}}

Acessada ao clicar no nome de usuário na parte superior da tela à
direita, perto do link de logout.

Página pessoal contendo suas informações, assim como links para Editar
dados pessoais \ref{participante-editar}, Alterar Senha
\ref{alterar-senha}, Certificados \ref{certificados}.

\subsubsection{Editar dados pessoais \label{participante-editar}}

Mapeada como:
\texttt{/participante/editar {[}application/ParticipanteController.editar{]}}

Edição de dados pessoais e de redes sociais.

\subsubsection{Alterar Senha \label{alterar-senha}}

Mapeada como: \texttt{/participante/alterar-senha}

\texttt{{[}application/ParticipanteController.alterar-senha{]}}

Requer senha antiga, nova senha e repetição da nova senha.

\subsubsection{Certificados \label{certificados}}

Mapeada como: \texttt{/participantes/certificados}

\texttt{{[}application/ParticipanteController.certificados{]}}

Lista com certificados de participação e que foi palestrante.
